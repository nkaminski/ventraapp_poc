
%% bare_conf.tex
%% V1.3
%% 2007/01/11
%% by Michael Shell
%% See:
%% http://www.michaelshell.org/
%% for current contact information.
%%
%% This is a skeleton file demonstrating the use of IEEEtran.cls
%% (requires IEEEtran.cls version 1.7 or later) with an IEEE conference paper.
%%
%% Support sites:
%% http://www.michaelshell.org/tex/ieeetran/
%% http://www.ctan.org/tex-archive/macros/latex/contrib/IEEEtran/
%% and
%% http://www.ieee.org/

%%*************************************************************************
%% Legal Notice:
%% This code is offered as-is without any warranty either expressed or
%% implied; without even the implied warranty of MERCHANTABILITY or
%% FITNESS FOR A PARTICULAR PURPOSE! 
%% User assumes all risk.
%% In no event shall IEEE or any contributor to this code be liable for
%% any damages or losses, including, but not limited to, incidental,
%% consequential, or any other damages, resulting from the use or misuse
%% of any information contained here.
%%
%% All comments are the opinions of their respective authors and are not
%% necessarily endorsed by the IEEE.
%%
%% This work is distributed under the LaTeX Project Public License (LPPL)
%% ( http://www.latex-project.org/ ) version 1.3, and may be freely used,
%% distributed and modified. A copy of the LPPL, version 1.3, is included
%% in the base LaTeX documentation of all distributions of LaTeX released
%% 2003/12/01 or later.
%% Retain all contribution notices and credits.
%% ** Modified files should be clearly indicated as such, including  **
%% ** renaming them and changing author support contact information. **
%%
%% File list of work: IEEEtran.cls, IEEEtran_HOWTO.pdf, bare_adv.tex,
%%                    bare_conf.tex, bare_jrnl.tex, bare_jrnl_compsoc.tex
%%*************************************************************************

% *** Authors should verify (and, if needed, correct) their LaTeX system  ***
% *** with the testflow diagnostic prior to trusting their LaTeX platform ***
% *** with production work. IEEE's font choices can trigger bugs that do  ***
% *** not appear when using other class files.                            ***
% The testflow support page is at:
% http://www.michaelshell.org/tex/testflow/



% Note that the a4paper option is mainly intended so that authors in
% countries using A4 can easily print to A4 and see how their papers will
% look in print - the typesetting of the document will not typically be
% affected with changes in paper size (but the bottom and side margins will).
% Use the testflow package mentioned above to verify correct handling of
% both paper sizes by the user's LaTeX system.
%
% Also note that the "draftcls" or "draftclsnofoot", not "draft", option
% should be used if it is desired that the figures are to be displayed in
% draft mode.
%
\documentclass[12pt,conference]{IEEEtran}
% Add the compsoc option for Computer Society conferences.
%
% If IEEEtran.cls has not been installed into the LaTeX system files,
% manually specify the path to it like:
% \documentclass[conference]{../sty/IEEEtran}



% *** PDF, URL AND HYPERLINK PACKAGES ***
%
\usepackage{url}
% url.sty was written by Donald Arseneau. It provides better support for
% handling and breaking URLs. url.sty is already installed on most LaTeX
% systems. The latest version can be obtained at:
% http://www.ctan.org/tex-archive/macros/latex/contrib/misc/
% Read the url.sty source comments for usage information. Basically,
% \url{my_url_here}.


% *** Do not adjust lengths that control margins, column widths, etc. ***
% *** Do not use packages that alter fonts (such as pslatex).         ***
% There should be no need to do such things with IEEEtran.cls V1.6 and later.


% correct bad hyphenation here
\usepackage{listings}
\usepackage{color}
\usepackage{hyperref}
\definecolor{dkgreen}{rgb}{0,0.6,0}
\definecolor{gray}{rgb}{0.5,0.5,0.5}
\definecolor{mauve}{rgb}{0.58,0,0.82}

\lstset{frame=tb,
  language=Python,
  aboveskip=3mm,
  belowskip=3mm,
  showstringspaces=false,
  columns=flexible,
  basicstyle={\small\ttfamily},
  numbers=none,
  numberstyle=\tiny\color{gray},
  keywordstyle=\color{blue},
  commentstyle=\color{dkgreen},
  stringstyle=\color{mauve},
  breaklines=true,
  breakatwhitespace=true,
  tabsize=3 
}
\begin{document}
%
% paper title
% can use linebreaks \\ within to get better formatting as desired
\title{Practical Exploits against the Integrity of Metra Mobile Tickets}

% author names and affiliations
% use a multiple column layout for up to three different
% affiliations
\author{\IEEEauthorblockN{Nash Kaminski}\\
nashkaminski@kaminski.io\\
Revision 1.1
}


% make the title area
\maketitle


\begin{abstract}
In this paper, a series of practical exploits against multiple data integrity vulnerabilities present in the Ventra mobile application are discussed. All exploits effect the current production version of the application unless noted otherwise and do not require any modifications to the application or the system software of the target device.
\end{abstract}
% IEEEtran.cls defaults to using nonbold math in the Abstract.
% This preserves the distinction between vectors and scalars. However,
% if the conference you are submitting to favors bold math in the abstract,
% then you can use LaTeX's standard command \boldmath at the very start
% of the abstract to achieve this. Many IEEE journals/conferences frown on
% math in the abstract anyway.

% no keywords



\section{Introduction}
As a daily Metra commuter and former computer engineering student at the Illinois Institute of Technology, I was significantly interested in the internal workings of the Ventra application pretty much from the day it was released. Specifically, I was curious as to the inner workings of a few features of the app in particular, such as those involving Metra mobile tickets. Even though the app was claimed to be secure, I did not see a logical means, given how the app operates for such functionality to be truly secure. After some investigation into such, I have discovered a series of serious vulnerabilities with varying difficulty of exploitation within such application that allow for the violation of the integrity of Metra mobile tickets and the compromise of user-entered data. None of the vulnerabilities presented require system level modifications, jailbreak, or root access to the target device or any modifications to the Ventra application itself and may be successfully executed as long as one controls the network traffic flows to and from the target device.

\section{Ventra Application Mobile Ticket Integrity}

\subsection{Impact}
Allows for limited-use tickets such as the one-way and ten-ride tickets to be used an infinite number of times. \emph{Mitigated 6/24/16 in release 1.2.1.39}

\subsection{Description}
The Ventra mobile application renders mobile tickets stored locally on the device prior to being able to successfully update the remaining number of uses for a given ticket on the remote server. Therefore, the following procedure can be followed to activate a limited use ticket a theoretically infinite number of times:
\begin{enumerate}
        \item Install(if necessary) and run Ventra application.
        \item If ticket is owned under a Ventra account, sign in to the account.
        \item Ensure ticket is displayed under "My Metra Tickets". If not, sync manually using the button at the top right.
        \item Activate airplane mode on the device.
        \item Use the mobile ticket(s).
        \item If the device is an Android device, go Settings $\rightarrow$ Apps $\rightarrow$ Ventra and choose "Clear Data".
        \item If the device is an iOS device, uninstall the Ventra application.
        \item Disable airplane mode.
        \item Observe that when the application is reopened (or in the case of iOS, reinstalled) that the use counter has \emph{not} decremented.
\end{enumerate}
If this procedure is followed properly, the device will render the ticket that it has stored locally, but will be unable to notify the remote server of the ticket's use. While the app does appear to maintain a queue of requests that are to be made to the remote server, this queue can be cleared by clearing the app's database (Android) or reinstalling it (iOS). This prevents the application from communicating the ticket usage event to the remote endpoint. Therefore, the original number of ticket activations is restored upon resyncing with the Ventra account after the app has been reinstalled or its database cleared allowing for a theoretically infinite number of uses of a limited-use ticket. Later versions of the Ventra application appear to maintain a copy of this queue and the ticket state within an encrypted file on the device filesystem. However the key to this encrypted store appears to be present as a constant within the application code and therefore this measure is likely able to be circumvented without great difficulty.

\section{Ventra Application API Information Disclosure}

\subsection{Impact}
Allows any unauthenticated user to access 14 days of mobile ticket security codes.

\subsection{Description}
The remote API accessed by the Ventra mobile application located at \url{https://ventra.transitsherpa.com/v2/rider/sync} does not verify that the device ID presented in the request headers actually owns any tickets prior to providing daily security codes. Additionally, the request signature provided in the \emph{x-gs-signature} header of the request is not validated by the remote server. Any request sent with a recent timestamp and claiming to be sent by an application running on a supported Android/iOS version results in the remote server providing the user with 14 days of valid ticket security codes. A proof of concept is implemented as the main function of v\_client.py and can be run by running:
\begin{lstlisting}
python3 v_client.py
\end{lstlisting}

\section{Ventra API Spoofing}

\subsection{Impact}
Allows for the rendering of a valid Metra mobile ticket with all parameters arbitrarily defined.\\
Allows for limited-use tickets such as the one-way and ten-ride ticket to be used an infinite number of times.

\subsection{Description}
This is by far the most challenging to exploit of the vulnerabilities presented and is partially dependent on the previous vulnerability for a source of valid mobile ticket security codes. However, such allows for a user to cause the Ventra mobile application to render \emph{any mobile ticket, regardless of what tickets, if any, the user actually owns} via the impersonation of the remote API endpoint accessed by the Ventra mobile application as well as compromise the integrity of some or all user entered data. While SSL is used to secure the communication between the Ventra application and its remote endpoint, no certificate validation is performed. As a result, the confidentiality and integrity of all data transacted to and from the remote API endpoint is compromised. One may in turn leverage either control over the WiFi network that the target is connected or Android and iOS's built in VPN functionality in order to gain control over the traffic flow between the application and backend server via either DNS modification and/or IP routing changes. Given control over such network flow, this allows the user or a malicious actor to intercept and/or impersonate the API endpoint accessed by the application at \url{https://ventra.transitsherpa.com/v2/rider}. No additional client-side verification of the data is performed by the Ventra application. A proof of concept server implementing a subset of the backend endpoints is located in
\begin{lstlisting}
app_server.py
\end{lstlisting}
and may be used to send responses to the app that define arbitrary Metra mobile tickets. Valid security codes can also be sourced using the previous vulnerability, such that the generated tickets appear completely genuine.

\section{Conclusion and Recommendations}
All of these vulnerabilities can enable the counterfeiting of mobile tickets as well as compromise of data transacted via the application if leveraged by a malicious actor. Therefore, these flaws should be mitigated or patched as quickly as possible. With regards to addressing these flaws, I would advise properly implementing request signature verification in the web API, as well as only providing security codes to devices which actually own valid mobile tickets. Serial numbers should be made unique to each ticket and the QR code containing such should be periodically audited. Tickets should only be rendered after the device has successfully communicated the ticket's use to the backend server. If rendering tickets without a data connection is absolutely necessary, such tickets should display a prominent notice and have their QR codes audited to ensure that their use is recorded. HTTPS certificates should always be validated and the implementation of certificate pinning should be considered, such that the user may not install his/her own certificate onto the device and regain the ability of spoofing the backend API server. Together, the implementation of either these specific measures or functionally similar measures will make manipulation of the Ventra application as well as the potential compromise of ticket authenticity or user data far more difficult.

\section*{Note}
This paper was written using the IEEE \LaTeX\ style found here: \href{http://www.ieee.org/conferences_events/conferences/publishing/templates.html}{IEEE - Manuscript Templates for Conference Proceedings}

% that's all folks
\end{document}


